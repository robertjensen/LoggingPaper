In every experimental laboratory the need for acquisition and subsequent
logging of data is of essential importance. The acquired data from a given
experiment is typically stored locally on the laboratory computer. The data is
often acquired in highly specialized proprietary software in closed proprietary
formats integrated with the hardware by the manufacturer. When working with
different experimental systems across several computers with different software
suites, operating systems etc. the exchange of acquired data is cumbersome and
difficult due to the nature of proprietary formats.

To avoid difficult data exchange, simplify data processing and integrate with
experimental equipment an open source software platform is
necessary\cite{Benn2009,Murray2011,So2007}. To accommodate these demands we
have written an open source platform for saving, logging, treating and
visualizing experimental data. The entire platform is written in Python, PHP
and HTML/CSS which is integrated with a MySQL server where the acquired
experimental data is stored. The Python language was primarily chosen due to
its tight integration with scientific packages which makes data analysis and
treatment more convenient\cite{Cahn2007}. PHP and HTML/CSS is used to display
data in standardized formats suitable for web browsers and process user input
the user. The combination of PHP and HTML/CSS to store data and process user
input from webpages is very flexible and has proven successful in other parts
of the scientific community\cite{Crane2008}.

For storage of acquired data a centralized server has been implemented. Storing
all acquired experimental data in a database on a centralized server running
source software has a number of attractive features. Firstly, by storing data
on a centralized server, backup of all experimental data is enormously
simplified. Secondly, the data is stored in a standardized and open formats
which allows easy export of the data to any platform, open format or software
source making data exchange across different platforms simpler. Thirdly, by
open sourcing the code on the server collaboration between several different
groups is possible hence increasing the number of developers to optimize the
code and further increase functionality. 

A centralized storage of data can be accomplished in many ways. However, in
experimental laboratories where large amounts of data are recorded a database
is an obvious choice for storage. As implementation we have chosen MySQL due to
its GNU General Public License, simplicity, fast performance, flexibility and
scalability.

Here we present a totally open source system consisting of data acquisition,
storage in a MySQL database and a comprehensive display module including simple
data treatment algorithms licensed as open source software.
