In every experimental laboratory the need for acquisition and subsequent
logging of data is of essential importance. Typically, logbooks, either in
electronic or paper form, is used to organize large amounts of data and note
down experimental parameters. The acquired data from a given experiment is
typically stored locally on the laboratory computer. The data is often acquired
in highly specialized proprietary software in proprietary formats. When working
with different experimental systems across several computers with different
software suites, operating systems etc. the exchange of acquired data is
cumbersome and difficult due to the nature of proprietary formats.

To avoid difficult data exchange and to further easy data processing and
integration with contingent equipment an open source software platform is
necessary\cite{Benn2009,Murray2011,So2007}. To accommodate these demands we
have written the entire system in Python, PHP and HTML/CSS integrated with a
MySQL server. Python was primarily chosen due to its tight integration with
scientific packages which makes data analysis and treatment
easier\cite{Cahn2007}. PHP and HTML/CSS is used to display the data to the user
in standardized formats which is suitable for web browsers. This approach is
very flexible and it used in other parts of the scientific
community\cite{Crane2008}.

A solution to simplify and make data exchange easier across several platforms
is to store all acquired experimental data on a centralized server running open
source software. This has a number of attractive features.
Firstly, by storing data on a centralized server backup of all experimental
data enormously simplified. Secondly, the data is stored in a standardized and
open format which allows easy export of the data to any platform or software
source.

The access to a centralized point of storage for experimental data also allows
for real-time continuous logging of various parameters used as indicators for
system health. Critical system parameters can hence be monitored and used to
trigger alarms when these fall out of specified ranges. Furthermore, the
logging of experimental setup parameters continuously enables access to these
values at any previous point in time.

A centralized storage of data can be accomplished in many ways. However, in
experimental laboratories where large amounts of data are recorded a database
is an obvious choice for storage. Here we present a totally open source system
consisting of data acquisition, storage in a MySQL database and a comprehensive
display module including simple data treatment algorithms which is open source
software.

