As mentioned in section~\ref{data_storage} it is both possible and indeed very
easy to access data directly from the database using either direct SQL-
statements or programming that will give access to the database. However, even
though this can be a good strategy for final publishable data-analysis, in
practice often one simply needs a quick look at data to see if everything
perform as expected. For this kind of data analysis, it would be highly
impractical to write small custom pieces of software each time. For this reason
we have developed a framework for visualization of the stored data that allows
quick and easy access to all data. The framework includes support for basic
data treatment as well as the option to option to plot several data sets at the
same time for easy comparison of results.

From the beginning, the goal for the visualization module has not been to
provide plots that have publication quality, since this is nearly impossible to
do in a rather simple plotting backing. Instead we aim for what we call
``supervisor-quality`` which means that the plots should be good enough that
they can be used for all kinds of everyday use, including presentations at
group meetings, etc. To achieve this we use frontend/backend type of topology
with the code used to handle user input being handled in HTML and PHP and thus
allows are user input to be performed from a web-browser. The backend consists
mainly of an open source plotting library and some interface code to feed data
into the plotting code and to setup preferences specific to each experimental
setup. The backend is mainly written in Python\cite{python} and the
configuration as done as an XML-file for each setup. The system is flexible
towards to choice of plotting library which is a great advantage since
different plotting libraries might be optimal for different tasks. At the
moment all plotting is performed by matplotlib\cite{matplotlib}. As a testimony
of the flexibility of the setup, we have moved to matplotlib from our original
backend, JpGraph\cite{jpgraph}, without significant changes to the backend
code. Currently, we are even investigating moving some of the display code to
the JacaScript based dygraphs\cite{dygraphs}, since this backend allows for
real-time manipulation of the data using a mouse, even though the graph is
still presented through an ordinary webbrowser.
