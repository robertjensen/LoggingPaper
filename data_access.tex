As mentioned in section~\ref{sec:data_extraction} it is both possible and very
easy to access data directly from the database using either direct SQL
statements or programming. This is very practical for the cases where it is
desired to perform data treatment on the data or to produce high quality
graphs. However, most of the time it is sufficient to simply look at the data
and possibly perform light data treatment. For this kind of data analysis it
would be highly impractical to write small custom pieces of software for each
kind of data that the users wish to look at. For this reason we have developed
a framework for visualization of the stored data that allows quick and easy
access to all data. The framework includes support for basic data treatment as
well as the option to plot several data sets at the same time for easy
comparison of results and to export the data to local files.

The goal for the visualization module has not been to provide high quality
plots since this is a large task that is best solved with either dedicated
software suites or custom scripts. Instead, the quality of the plots was
targeted such that it is sufficient for different kinds of everyday use
including presentations at informal group meetings, as starting point for
discussion of data, quick data comparison etc. To realize this form of
visualization a frontend/backend topology has been implemented. The code that
handles user input is retrieved in HTML and PHP thus allowing user input to be
acquired from a web browser. The backend consists of an open source plotting
library, interface code to feed data into the plotting code and plot
preferences specific to each experimental setup. The backend is mainly written
in Python and the configuration read from a XML file unique to each setup. The
preferences file contain a settings section for each different kind of plot.
Below is shown an example of a continuously logged pressure:

\begin{verbatim} 
<!-- PRESSURE --> 
<graph type='pressure'>
  <query>SELECT unix_timestamp(time), pressure FROM pressure_SETUP
  where  time between " {from}" and "{to}" order by time</query>
  <ylabel>Pressure / Torr</ylabel>
  <title>Pressure in {setup}</title>
  <default_yscale>log</default_yscale>
  <default_xscale>dat</default_xscale>
</graph>
\end{verbatim}

The system is flexible towards choice of plotting library which is a great
advantage since different plotting libraries are optimal for different tasks.
At the moment all plotting is performed by matplotlib\cite{matplotlib}.  As a
testimony to the flexibility of the setup we have moved from our original
backend based on JpGraph\cite{jpgraph} to a backend based on matplotlib without
significant changes to the backend code. Currently, we are investigating moving
some of the display code to the JavaScript based dygraphs\cite{dygraphs} thus
producing a dual display system since this plotting library allows for real-time 
manipulation of the data using a mouse through an ordinary web browser.
