As mentioned in section~\ref{data_storage} it is both possible and very easy to
access data directly from the database using either direct SQL statements or
programming that will give access to the database. However, even though this
can be a good strategy for final publishable data analysis, in practice often
one simply needs a quick look at data to see if everything performs as expected.
For this kind of data analysis it would be highly impractical to write small
custom pieces of software for each user. For this reason we have developed a
framework for visualization of the stored data that allows quick and easy
access to all data. The framework includes support for basic data treatment as
well as the option to option to plot several data sets at the same time for
easy comparison of results.

From the beginning the goal for the visualization module has not been to
provide plots that have publication quality since this is nearly impossible in
a simple plotting backing. Instead we aim for a quality where the plots are
good enough to be used for all kinds of everyday use including presentations at
group meetings, as starting point for discussion of data etc. To achieve this
we use a frontend/backend type of topology where the code used to handle user
input is retrieved in HTML and PHP thus allowing user input to be acquired from
a web browser. The backend consists of an open source plotting library,
interface code to feed data into the plotting code and setup preferences
specific to each experimental setup. The backend is mainly written in
Python\cite{python} and the configuration read from a XML file unique to each
setup. The system is flexible towards choice of plotting library which is a
great advantage since different plotting libraries might be optimal for
different tasks. At the moment all plotting is performed by
matplotlib\cite{matplotlib}. As a testimony of the flexibility of the setup, we
have moved to matplotlib from our original backend, JpGraph\cite{jpgraph},
without significant changes to the backend code. Currently, we are even
investigating moving some of the display code to the JavaScript based
dygraphs\cite{dygraphs}, since this plotting library allows for real-time
manipulation of the data using a mouse even though the graph is still
presented through an ordinary web browser.
