The existence of this data system is paramount to several automation tasks
in our lab. Below a few examples of the use of the data system for
these purposes are described.

\subsection{Sample cleaning}
In the field of catalysis, one direction of research concerns gas
reactivity studies on the surface of single crystal samples under
Ultra High Vacuum (UHV)\fixme{Probably has been introduced before}
conditions ($\sim10^{-13}$\,mbar).

Before experiments can be performed, the sample must be properly
cleaned to ensure that no contaminants are present on the surface. The
cleaning of the surface is achieved by running a number of cleaning
cycles. Each of these cycles can take up to several hours.% and may
%include sputtering (bombardment of the surface with Ar$^+$ ions to
%peel or a part of the surface), gas exposure and heating.
Before this task was automatized, it typically required simple manual
intervention 2-4 times during a 30 to 120 minute cycle. Obviously, this
was a suboptimal solution, since a lot of time was spent, with only
small time intervals to work on other things, before the next manual
intervention.

As it is apparent the atomization of this task lead to significant
time savings, since it allowed these cleaning cycles to e.g.\ be run in
the evening or at night. However, during the atomization process one
concerns that had to be addressed was how to monitor the status of the
cleaning and the equipment. This was easily accomplished by means of
the data logging system, where all relevant parameters can be
continuously logged and viewed with any of the data viewing options
mentioned in section \fixme{insert ref}. Obviously the cleaning
program it self is responsible for the safety of the system and for
shutting it down if it goes out of boundaries. But with the continuous
logging it is a simple task to add surveillance to the task that
alerts the user if it happens, in case it can be safely restarted
remotely.

\subsection{Experimentation over extended time periods}

Mini-reactor example. Bla bla more exhaustive search of parameter
space for experiments with large time spans. \fixme{Finish section}

\subsection{Cooling water alarms}
Possibly add section about the cooling water alarms. \fixme{Complete
  or remove this section}