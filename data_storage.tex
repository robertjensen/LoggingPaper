For storage of acquired data a centralized server has been implemented as
opposed to other non-centralized systems\cite{McIntosh2003}. Storing all
acquired experimental data in a database on a centralized server running open
source software has a number of attractive features. Firstly, by storing the
data on a centralized server, backup of all experimental data is enormously
simplified compared to backup of individual computers. Backup of experimental
data is, hence, from a user's point of view automatic as this can be performed
by routine jobs on the server. Secondly, the data is stored in a standardized
and open format which allows for easy export of the data to any platform, open
formats or software source. This makes data exchange across different platforms
immensely more simple. Thirdly, by open sourcing all of the code used to
visualize and treat data from the database, collaboration between several
different groups is possible thus increasing the number of developers to
optimize the code and further increase functionality.

A centralized storage of data can be accomplished in many ways. However, in
experimental laboratories where large amounts of data are recorded a database
is an obvious choice for storage. To keep the server backend simple a
relational database has been chosen. As specific implementation MySQL has been
chosen due to its GNU General Public License\cite{gpl}, simplicity, fast
performance, flexibility and scalability.

A system design of many highly decentralized clients all pushing data
continuously to a central MySQL server requires high server performance, high
uptimes as well as a flexible storage ensuring easy expansion of storage space
if needed. To ensure these properties of the system it has been implemented
with as few modification to the configuration of the standard server software
as possible, which helps to ensure that this central component can easily be
managed by the professional IT-staff at the department. It is important to
realize that while the clients can easily have the ''age diversity'' of
accumulated scientific equipment, be exposed to harsh conditions in the lab and
be managed by the scientific staff the server needs to be managed and handled
with all the care associated with a production environment server.

To protect against pollution of the various setups tables in the database each
client has its own username and password which is not part of the code
(typically it will be managed in the local ODBC-settings of the client). In
this way, interface code can flow back and forth between different setups
without the risk of one setup accidentally logging data to other setup's
tables.

For each of the setups that is connected to this system, the following table
structure is implemented. Each of the measurements that are continuously logged
has their own table, where the values are logged as a function of date and
time. All the specific measurements, such as spectra, scans and values
monitored over a well-defined time span are stored in two tables in the
database, a metadata table and a data table. In the metadata table \textbf{all}
metadata pertaining to the measurement is saved in one row. In the data table
xy-values and a measurement id is saved in one row per xy-value entry. The
metadata in the metadata table is then connected to the data points in the data
table by a unique id attached to each xy-value entry. This structure is a
compromise between space usage and simplicity as it leads to extra space being
used to store the measurement id once for every data point. This, on the other
hand, makes it possible and simple to store all the different kinds of xy-data
in just two tables.

\subsection{Data extraction} \label{sec:data_extraction}

The flexible nature of SQL allows one to extract data in many ways. Complete
datasets, data in a certain time interval for a continuously logged
measurements or data pertaining a specific measurements can be retrieved by
simple statements, 
\begin{verbatim}
 SELECT * FROM table_name
 SELECT * FROM table_name WHERE time BETWEEN {from} AND {to}
 SELECT * FROM table_name WHERE measurement={id}
\end{verbatim}
(where the entities marked by \{\} are variables that should be
replaced with specific values)

This data can then be handled using the programming environment most
comfortable for the user and can be used to perform automatic reporting, data
treatment or as input to scripts that will produce plots based on the data. SQL
also allows for very efficient data treatment directly from the SQL-server,
which can be very useful to get a quick overview of the acquired data, as
illustrated e.g.\ in section \ref{sec:morning_pressure}.

A further advantage of SQL servers is the standardization, which makes it easy
to change the choice of implementation, if it is wanted for some reason.
Several open source implementations of SQL-servers exists including Firebird,
PostgreSQL, Oracle, Mimer SQL.
