Our system-design of many highly decentralized clients all pushing data
continuously to a central server means, that this server must both high
perfomance, high uptimes as well as the a flexible storage to ensure that it is
always possible to add more space if so needed. To ensure these properties of
the system, we have deliberately chosen a system as simple as possible to avoid
unneccessary complications, and to ensure that this central component can be
easily managed by the professional IT-staff of the derpartement. It is
important to realized that even though it typically is not a problem if a
single client machine somewhare in the system is temporarily down (this can
happen for many reasons in a experimental lab), it is crucial that this
particular server is handled like a real production server.

To keep the server-side of things simple, we use a relational database, in our
case MySQL\cite{mysql} which is open source and provides extremely good
performance. This MySQL server contains the entire set of data of all
experimental setups, thus it is extremely easy to backup everything by simply
doing a dump of the database with regular intervals.

Since the database is only exposed to the local network, security is not a
really a concern, however to protect agains accidential pollution of the
various setups with irrelevant data when code is exchanged between setups, each
client has its own username and password which is not part of the code
(typically it will be managed in ODBC-settings) In this way, code can flow back
and forth between setup without the risk of one setup accidentially loggin data
to other setups tables.

